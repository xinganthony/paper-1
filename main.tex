%% ****** Start of file aiptemplate.tex ****** %
%%
%%   This file is part of the files in the distribution of AIP substyles for REVTeX4.
%%   Version 4.1 of 9 October 2009.
%%
%
% This is a template for producing documents for use with 
% the REVTEX 4.1 document class and the AIP substyles.
% 
% Copy this file to another name and then work on that file.
% That way, you always have this original template file to use.

\documentclass[aip, pop, preprint]{revtex4-1}
%\documentclass[aip,preprint]{revtex4-1}


% Include some handy packages
\usepackage{amssymb,amsmath,color}
\usepackage{graphicx}
%\usepackage{wrapfig}
%\usepackage{caption}
%\usepackage{subcaption}
%\usepackage{placeins}
%\usepackage{setspace}
\usepackage{xfrac}
%\usepackage{pdfpages}

%\draft % marks overfull lines with a black rule on the right

\begin{document}
\graphicspath{{figures/}{plots/}}

% Use the \preprint command to place your local institutional report number 
% on the title page in preprint mode.
% Multiple \preprint commands are allowed.
%\preprint{}

%\title{Investigation into Classical Ion Heat Transport in Enhanced-Confinement MST Plasmas using Integrated Forward Modeling} %Title of paper
\title{Classical ion heat transport in RFP plasmas}
% repeat the \author .. \affiliation  etc. as needed
% \email, \thanks, \homepage, \altaffiliation all apply to the current author.
% Explanatory text should go in the []'s, 
% actual e-mail address or url should go in the {}'s for \email and \homepage.
% Please use the appropriate macro for the type of information

% \affiliation command applies to all authors since the last \affiliation command. 
% The \affiliation command should follow the other information.

\author{Z.A. Xing}
\email[]{zaxing@wisc.edu}
%\homepage[]{Your web page}
%\thanks{}
%\altaffiliation{}
\affiliation{University of Wisconsin-Madison}
\author{M.D. Nornberg}
\affiliation{University of Wisconsin-Madison}
\author{J. Boguski}
\affiliation{University of Wisconsin-Madison}
\author{D. Craig}
\affiliation{Wheaton College}
\author{D.J. Den Hartog}
\affiliation{University of Wisconsin-Madison}
\author{K. McCollam}
\affiliation{University of Wisconsin-Madison}
\author{T. Nishizawa}
\affiliation{University of Wisconsin-Madison}



% Collaboration name, if desired (requires use of superscript address option in \documentclass). 
% \noaffiliation is required (may also be used with the \author command).
%\collaboration{}
%\noaffiliation

\date{\today}

\begin{abstract}

We report that classical ion heat transport modeling has successfully predicted the observed $T_i$ profile evolution in tearing suppressed RFP plasma in the Madison Symmetric Torus without invoking anomalous heating terms. The model incorporates all available diagnostic data to forward model T$_{i}$, which is then compared to charge exchange spectroscopy measurements. In tearing-suppressed RFP plasmas stochastic transport is greatly reduced and neoclassical effects on ions are small, allowing classical effects to become dominant. Monte Carlo modeling of neutral dynamic with DEGAS2 significantly lowers the estimated charge exchange loss as compared to previous studies via both lower estimates of core neutral density as well as finite neutral temperature. At the same time, further investigations of the density evolution in the tearing suppressed plasmas had found previously predicted inwards pinch flow associated with current drive as well as decreased the estimated convective loss. With the reduced loss terms in the model, the ion temperature in the core is no longer considered to be anomalously high in the tearing suppressed period and ion power balance is found to be driven by classical effects, especially compression heating, and charge exchange transport.


\end{abstract}

\pacs{}% insert suggested PACS numbers in braces on next line

\maketitle %\maketitle must follow title, authors, abstract and \pacs


\section{Introduction}
\label{sec:intro}

During standard operation of an RFP, closely spaced tearing resonant surfaces result in overlapping magnetic islands and stochastic transport, as well as periodic global magnetic reconnection referred to as sawtooth events. These tearing mode instabilities greatly limit the confinement achieved on the RFP\cite{Bodin1980Reversed-field-pinchReserarch, Sarff1995TransportPinch}. The Madison Symmetric Torus (MST) is able to operate with greatly reduced tearing mode activity via Pulse Poloidal Current Drive (PPCD), an inductive method of flattening the current profile\cite{Chapman2001ReducedPinch, Sarff1995TransportPinch}. During PPCD, $ \geq 10ms $ of tearing mode suppression are observed resulting in improved confinement. During these periods, T$_{e}$ rises significantly\cite{Chapman2001ReducedPinch}. 

The ion thermal transport characteristics during PPCD was not well understood. Ion temperature becomes decoupled with that of the electron as collisonality is reduced. Previous estimates of ion thermal loss terms, especially charge exchange and convective loss, pointed to anomalously high ion temperatures\cite{Fiksel2006Confinement, Wyman2007THEPLASMAS, BiewerThesis}. However, the source of the anomalous is not clear, as typical anomalous heat sources such as magnetic reconnection is suppressed during the PPCD period. Further adding to the confusion, core impurity ion particle diffusion in PPCD plasmas in MST have been shown to be largely classical \cite{Kumar12prl}. The unidentified anomalous heating mechanism casts uncertainty on the success of tearing suppression as a confinement strategy.

Neutral and sourcing research on other machines are becoming more common. 

In this paper, we show that by improving the modeling of neutral particles in MST coupled with the correct determination of inward pinch flow and it's effect on thermal transport, an classical model of ion thermal transport is sufficient to predict and explain the $T_i$ evolution observed. This work starts by describing the approach to investigate ion energy balance in PPCD; then moves on to the incorporation of Monte Carlo neutral simulation via DEGAS2 and its impact on the understanding of charge exchange loss; subsequently, presents the observation pinch flow and its effect on ion heat transport; and finally ends on results from comparing model predictions with observed $T_{i}$ profile evolution and peaking.

\section{Experimental Setup}

A model of classical heat transport is created to investigate the applicability of classical ion energy balance. The forward model predict the evolution of the 1-D $T_i$ profile by integrating input signals from a range of plasma diagnostics. These includes Thompson Scattering(TS) measurements of $T_e$, the Far-InfraRed Interferometer and Polarimeter(FIR) measurements of $n_e$ and Faraday rotation. Additionally, the model also includes equilibrium reconstruction via MSTFIT\cite{Anderson04}, as well as neutral modeling through DEGAS2. The impurity density is assumed to be constant in the period of interest based on previous measurements \cite{Kumar12pop,Nornberg18FST}. The model further uses an initial temperature profile from measurements. 



The forward model is constructed as a 1-D cylindrical approximation with the volume averaged minor radius ($ \rho_v $) of any given flux surface as the coordinate. The needed values are volume averaged over flux surfaces, since parallel transport is much faster than any of the perpendicular transport considered in this work.

The model consist of the following terms:

\begin{equation}\label{eqn:balance}
\frac{\partial}{\partial t}\left(\frac{3}{2}nkT_{i}\right) = P_{e-i} + P_{cond} + P_{flow} + P_{cx} % + P_{anomalous}
\end{equation}

To help constrain $P_{flow}$, particle flow continuity equation is also imposed.The importance of this constraint is discussed in section \ref{flow_effects}

\begin{equation}\label{eqn:cont}
\int_{S} \Gamma\cdot dA = S_{tot}-\frac{\partial N}{\partial t}
\end{equation}



$ P_{e-i} $ and $ P_{cond} $ are the collisional equilibration heating (ion-electron) and classical conduction (ion-ion) terms. These classical terms are straight forward to calculate from equilibrium reconstructions as follows:

\begin{align}\label{eqn:p_ei}
    P_{e-i} &= \frac{3}{2}n_i\nu^{i/e}(T_e - T_i)\\
    \nu^{i/e} &= 5.69\times10^{-27} \frac{\sqrt{m_i m_e}(q_i q_e)^2 n_e \Lambda}{(m_i T_e + m_e T_i)^{3/2}}\\
    P_{cond} &= \frac{1}{\rho}\frac{\partial}{\partial\rho}(\rho\kappa_{\perp}\nabla T_{i})\\
    \kappa_{\perp} &= \sqrt{2} \times 10^{-2}\frac{n_i T_i \nu_i}{m_i(\frac{q_i}{m_i} |B|)^2}
\end{align}

$P_{cx}$ is the charge exchange and $P_{flow}$ is the flow related term (convection, advection, and compression). These latter terms are more complex to determine and forms the majority of the work presented here. First, I will explore the importance of neutral modeling, especially in regards to the temperature of the neutrals. Then I will present the observation of inward pinch flow and it's effect on the model.

\section{Neutral Dynamics and Charge exchange}\label{neutral}

MST's neutral density is higher than what is typically found in diverted tokamaks. Charge exchange collisions causes thermal ions to becomes neutral and is "replaced" by the formerly neutral particle. As the neutral population is cool, this process represents a loss mechanism. In PPCD plasmas where other loss mechanism are reduced, charge exchange loss becomes dominant. There has been no previous attempt to systematically characterize the charge exchange term in MST PPCD plasmas, but rough estimation had shown that charge exchange loss is large and need an offsetting anomalous heating term even when tearing modes are suppressed\cite{BiewerThesis}.

On MST, $D_{\alpha}$ emission due to impact excitation and charge exchange is observed via an array of 13 silicon detectors. This is dominated by emission in the plasma edge where neutral density is high. As the core contributes negligible emissions, a simple inversion of line integrated measurements have difficulty capturing more than an upper bound of the core neutral density. Therefor, physics based modeling is needed to extend neutral profiles to the core, where it might have a strong effect on expected charge exchange loss and impurity state balance.

In previous estimates, the neutral fluid is assumed to be cold, therefore, if a thermal ion undergoes charge exchange it represents a total energy loss. This assumptions is incorrect. The mean free path of a room temperature neutral is very short, even in a typical edge plasma, where as thermal neutrals created via charge exchange have a much longer mean free path. Consequently, neutrals penetrating to the core are mostly generated near the mid-radius and therefore have a temperature comparable to mid-radius ions. At the same time, calculations concludes that a charge exchange neutral created in the core of the plasma, if traveling through core-like conditions, have a mean free path shorter than the minor radius, implying a fraction of such neutrals would undergo secondary ionization or charge exchange reactions.  At the same time, with a long mean free path compared to the minor radius ()$Kn \simeq 0.8$), the neutral species cannot be adequately treated as a fluid. 

\begin{figure}
	\centering
	\includegraphics[width = 1.\linewidth]{./plots/degas_neutral_n}
	\label{fig:DEGAS2_2d_density}
	\caption{Typical neutral density result from DEGAS2 show a rapid drop off towards the core. There is two notable asymmetry: the first where the Shafranov shift result in lower neutral density on the outboard side, and the second where the gas puffing that occurs before the PPCD period leaves a residual up/down asymmetry.}
\end{figure}%

DEGAS2, a 2-D Monte Carlo simulation that produces  neutral density and temperature profiles, was incorporated to model core neutral profiles. DEGAS2 use  $ T_{e} $ and $ n_{e} $ profiles as input, and it tracks and tallies charge exchange, ionization, recombination, and molecular disassociation reactions, as well as associated particle and heat flow of test particles. Synthetic $ D_{\alpha} $ diagnostic are created within DEGAS2, and they are used to fit experimental measurements of the same, using boundary source rates as fitting parameters. The precise mechanism behind this neutral source, such as recycling, is outside of the scope of this work. A three surface source geometry, each having an independent source rate is used, consisting of an outboard limiter, a pumping duct region (bottom 45\textdegree) and the rest of the wall. The outboard limiter is singled out for special attention due to the Shafranov shift causing the last closed flux surface to strike the outboard limiter rather than the inboard. The pumping duct being a separate source was found to be needed to improve the fit quality in PPCD conditions. The sources, and their contributions to the plasma are assumed to be linearly independent as neutral-neutral interaction is negligible. An example of the result of the DEGAS simulation for a single shot and time is shown as fig \ref{fig:DEGAS2_2D}.

\begin{figure}
	\centering
	\includegraphics[width = 1.\linewidth]{./plots/degas_neutral_t}
	\label{fig:DEGAS2_2d_temp}
	\caption{Neutral modeling results also show a corresponding 'rise' in temperature as one moves towards the core. This produces a charge exchange loss profile that is both lower and more hollow than previously thought.}
	\label{fig:DEGAS2_2D}
\end{figure}

DEGAS2 simulations are evaluated at 0.5ms intervals despite the forward model propagates at 1$\mu$s steps.  This timing is chosen due to both the  computational cost of the Monte Carlo simulations, and the frequency at which $ T_e $ measurements, a key input, are available via TS. The 2-D results are incorporated into the 1-D ion thermal model through flux surface averaging.

DEGAS2 modeling shows $\frac{T_{neutral}}{T_{i}} \simeq 0.7$[[TODO: recheck this value!!]] in the core, which combined with lower $ n_{neutral} $ leads to lower charge exchange heat loss than previous estimates. Further, the charge exchange loss is found have a hollow profile, and is at a level that is broadly consistent with the temperature evolution of the ions without having to invoke additional anomalous heating terms during PPCD periods. 

\section{Flow related effects on heat}\label{flow_effects}

In standard MST plasmas, $P_{flow}$ account for $ ~10\% $ of electron heat-loss \cite{BiewerThesis}, and during PPCD it may become more important as other loss terms are suppressed. Calculating $P_{flow}$ involves estimating the ion particle flux.  However, no diagnostic measurements of $n_i$ in MST is available, therefore, $n_i$ is inferred indirectly through the evolution of $n_e$ combined with previous work on characterizing the impurity content in PPCD\cite{Kumar12pop,Nornberg18FST}.

MSTfit uses the 11-chord line integrated $n_e$ measurement from the FIR interferometer to reconstruct density profile. Measurements show clear rise of core $n_{e} $in the early half of the PPCD period, ending around 16.5ms. It is important to determine the nature of this density rise, as the temperature of these "new" ions would have a significant effect on the model predictions.

The DEGAS2 simulations characterizing the neutral dynamics also provides a tally of the electron source rate due to the ionization of the neutrals. This source primarily concentrate in the edge and is too low in the core to account for the density rise in the core(fig \ref{fig:ne_change}). The further ionization of impurities due to increasing temperature is another possible source. Impurity contribution to the electron density rise is estimated using previous measurements, including carbon, aluminum, oxygen and boron, which were assumed to be in coronal equilibrium. Their contribution to the electron source rate is calculated from the change of charge state balance. The result suggests impurity contribution to $n_e$ rise is insignificant.


\begin{figure}
	\centering
	\includegraphics[width=0.95\linewidth]{./plots/dndt_at14-1}	
	\caption{Electron density change show that the $n_e$ source terms are concentrated at the edge, and thus core density rise needs to be accounted for by flux.}
	\label{fig:ne_change}
\end{figure}

\begin{figure}
	\centering
	\includegraphics[width=0.95\linewidth]{./plots/flux_comp}
	\caption{Estimated flux values. The calculated $ E\times B $ flux is in red. It goes to zero at the edge as the density goes to zero.}
	\label{fig:flux_compare}


\end{figure}

The calculated electron source rates being insufficient to account for the density rise, there must be an inward particle flux. The particle flux ($\Gamma_{obs}$) needed to satisfy the continuity equation (\ref{eqn:cont}) is then calculated. However, we seek to confirm the theoretical plausibility of an inward pinch by considering the evolution of the magnetic equilibrium.

MST's $\vec{E}$ diagnostic measures plasma potential and does not constrain the radial $\Gamma_{\vec{E} \times \vec{B}}$. Instead, the slow time-scale E field is estimated using the evolution of reconstructed B field from edge flux coil and FIR polarimetry data. The MSTfit equilibrium reconstruction outputs reconstructed flux values and flux surfaces in 2D which enables 

meaning that $ E_r $ cannot be reconstructed, but as the $ E \times B $ flux of interest is in the radial direction only $ E_{pol} \text{ and} E_{tor} $ need to be estimated. In particular $E_{pol}$ can be calculated through:

\begin{align}
E_{pol}(\rho_v) & = \int_{0}^{\rho_v}\rho_v' \frac{dB_{tor}}{dt} d\rho_v'
\end{align}

%where $\rho_v$ is the flux surface averaged minor radius.

$E_{tor}$ is slightly more complicated, as it is in reality a function of major radius. To incorporate into the 1-D approximation, $E_{tor}(R, Z=0)$ along the Z = 0 plane is determined as a function of major radius (R) through equation (\ref{eqn:E_tor}) and boundary condition (\ref{eqn:E_tor_bc}). The results and then averaged according to flux surface. 

\begin{align}
E_{tor}(R) & = -\frac{1}{\int_{R_0 - a}^{R_0 + a} R' \frac{\partial B_{pol}}{\partial t} dR'} - B_{tor}(R_0 - a) \label{eqn:E_tor}\\
E_{tor} (R_0 - a) & = \frac{- V_{poloidal\ gap} }{2 \pi R}\label{eqn:E_tor_bc}\\
E_{tor}(\rho_v) & = \frac{1}{2}(E_{tor}(R_{in}(\rho_v)) + E_{tor}(R_{out}(\rho_v))
\end{align}
where R refers to the major radius, and r refers to the minor radius from the magnetic axis.

Since the $ \frac{dB}{dt} $ term cannot be evaluated directly, sequential MSTfit reconstructions, 0.5ms apart is used to provide $ B_{pol} $ and $ B_{tor} $ information from which numerical derivative taken.

$E_{pol}$ is integrated with the boundary condition that the E field is zero at the magnetic axis. The measured voltage at the inboard poloidal gap provides the boundary condition of the $ E_{tor} $ integration. The results show it to be relatively stable in the core, but changes direction in the edge as time progresses. This reversal correlates with current drive being exhausted in the edge. Note that the edge $\vec{B}$ field reveres occurs earlier.

From there the radial particle flux is calculated as:

\begin{equation}
\Gamma_{\vec{E} \times \vec{B}} = n_{e} \frac{E_{pol}B_{tor} - E_{tor}B_{pol}}{B^2}
\end{equation}

This is compared to $\Gamma_{implied}$ previously calculated from the continuity equation in figure \ref{fig:flux_compare}. For the core to about the mid radius, the estimated $\Gamma_{\vec{E} \times \vec{B}}$ tracks the flux needed to account for density change well. However, farther in the edge, there is residual flow outwards since the neutral ionization rates are faster than density growth, thus particle loss is need to balance the continuity equation. This is likely cause by anomalous mechanisms such as residual tearing mode or drift wave activity.

Importantly the $ E_{tor} $ reversal during the PPCD period causes the estimated $ \Gamma_{\vec{E} \times \vec{B}} $ flow to cease. This echoes MHD calculations by J. Reynolds that found the $ E \times B $ flow would cease as the axial E field is reduced and reversed\cite{ReynoldsThesis}.

It is possible to calculate, in lab frame, particle flow's contribution to the heat balance as well as thermodynamic compression:

\begin{align}
P_{flow} & = -\frac{3}{2}\vec{V}\cdot\vec{\nabla}p - \frac{5}{2}p\cdot\vec{\nabla}\cdot\vec{V}\\
& = -\frac{3}{2}\vec{\nabla}\cdot(nkT\vec{V}) - p\cdot\vec{\nabla}\cdot\vec{V}\label{eq_flux_terms}
\end{align} 

Here pressure $ p = nkT $. In the scope of this work, it is useful to breakout the thermodynamic work due to compression $  ( p\cdot\vec{\nabla}\cdot\vec{V} ) $ separately from the conservation terms for presentation, like in equation (\ref{eq_flux_terms}). This is incorporated into the 1-D model by assuming $ \vec{\nabla}\cdot\vec{V}\vert_{\theta, \phi} = 0 \text{ and } \vec{\nabla}p\vert_{\theta, \phi} = 0 $.


\begin{figure}
	\centering
	\includegraphics[width=1.\columnwidth]{./plots/dedt_june18-1.png}
	\caption{Flow effects are the most significant.\label{fig:dedt_plot}}
\end{figure}

Using the total implied flux (including $ E \times B $ and anomalous), the flow contribution to heat balance was calculated and found to be the largest term in the core of the plasma. Breaking it down further, the notable observation is that the compression heating is larger than the equlibration heating and is the largest heating term in the core. The conservation term is larger, but does not cause temperature rise as it represents the thermal energy being carried by the pinching particles. The temperature in the core is predicted to rise slightly while in the gradient region it would decrease as the colder ions flows in.

\section{Results and discussion}

To assess the applicability of the model. It's predictions are compared to measurements of $T_{C^{6+}}$ using Charge Exchange Recombination Spectroscopy\cite{DenHartog1994ADynamics,DenHartog1994ADynamics}. CHERS measurements of C+6 temperature is used as a proxy for majority ion temperature for model comparisons since $\tau_{C^{+6}-D^{+}}$is small compared to the time scale of analysis, it is taken as a proxy of majority ion temperature as direct measurement is not available\cite{Reardon03}, and the primary drive that brings the two temperatures out of equilibrium are sawtooth events which are suppressed in PPCD plasmas\cite{Fiksel2009Mass-dependentPlasma}. To begin the analysis, a ion temperature profile is needed to initialize the model. This is taken from CHERS measurements at 12ms. Since MST's CHERS system can only measure temperature at one location at a time, the profile is constructed from an ensemble of measurements at different locations fitted to a two power profile across the radius. 12ms being the earliest time that tearing mode is significantly reduced by PPCD. 

\begin{figure}[!h]
	\centering
	\includegraphics[width=0.95\columnwidth]{./plots/temp_comp_ext}
	\caption{Temperature prediction vs observations. The classical thermal transport model is able to successfully prediction $T_i$ evolution during PPCD up to $\rho /a = 0.55$. Note that model does not initialize at exactly the observed value at a given chord as it is fit to a smooth profile in the radial direction.\label{fig:comp}}
\end{figure}

The model is initialized with $T_i$ profile obtained by fitting CHERS measurement at 12ms. The time is chosen for being the earliest time that PPCD consistently suppresses tearing mode activity. The model is then allowed to propagate until 19ms, and it's prediction of $T_i$ is compared to the temperature measurement from an ensemble of plasma shots where PPCD was successful. The result of this comparison is show in fig. \ref{fig:comp}. Notably the model was able to predict the maintenance and slight rise in temperature without invoking any anomalous heating. The model, however, performed poorly in the $\rho /a \approx 0.75$ region.

Previous estimations that resulted in the $T_i$ being considered anomalously high suffered from the poor estimation of the neutral interactions\cite{Fiksel2006Confinement}. The calculation used was based on simple inversions of the $D_{\alpha}$ emission, and it impact the ion transport calculations in two ways. First, the calculation of the CX loss term, where the overly high estimate of core $n_n$, as well as the assumption of total loss (ie. cold neutrals) resulted in overly high estimates of CX loss. Secondly, the overly high estimates of the source term led to overly high estimates of outward particle flux, and therefore overly high convective losses. The proper accounting of the neutral dynamics is therefore crucial to the proper accounting of thermal transport. Recent development in the Tokamak world also moving towards more detailed evaluation of the neutrals' impact on transport calculations by coupling neoclassical PIC code to Monte Carlo neutral transport code\cite{Stotler2013PedestalCode}.

Neoclassical transport have been ignored in the model since they are expected to be small for ions in the RFP. Despite the deuterium expected to be in the banana regime\cite{Kumar12pop} in MST, the neoclassical transport is smaller than classical as $\frac{q^2}{\epsilon^{3/2}} \leq 0.2$ everywhere in MST. Another effect that is ignore in this model is the turbulence transport, and this likely causes the inability for the classical model to predict $T_i$ evolution at $\rho /a \approx 0.75$. Recent studies into turbulence transport associated with drift wave turbulence in the edge of MST plasma broadly coincides with this conclusion\cite{NishizawaPRL}.

Further, the successful modeling of tearing suppressed RFP plasma  

We observe strong ion heating associated with magnetic reconnection events in reversed-field pinch (RFP) plasmas\cite{Gangadhara08}, and it have much faster characteristic time than ion-electron thermal equlibration. Several mechanisms have been proposed to account for this anomalous heating including Landau damping, electron cyclotron damping, stochastic heating, and ion cyclotron damping\cite{Tangri08}. Further, isolation of the anomalous heating associated with sawtooth reconnection events, from possible other anomalous heating that is ongoing during the plasma discharge, is an interesting topic of investigation. 

This work is supported by the U.S. Department of Energy, Office of Science, and Office of Fusion Energy Sciences under Award Number DE-FC02-05ER54814.


% If in two-column mode, this environment will change to single-column format so that long equations can be displayed. 
% Use only when necessary.
%\begin{widetext}
%$$\mbox{put long equation here}$$
%\end{widetext}

% Figures should be put into the text as floats. 
% Use the graphics or graphicx packages (distributed with LaTeX2e).
% See the LaTeX Graphics Companion by Michel Goosens, Sebastian Rahtz, and Frank Mittelbach for examples. 
%
% Here is an example of the general form of a figure:
% Fill in the caption in the braces of the \caption{} command. 
% Put the label that you will use with \ref{} command in the braces of the \label{} command.
%
% \begin{figure}
% \includegraphics{}%
% \caption{\label{}}%
% \end{figure}

% Tables may be be put in the text as floats.
% Here is an example of the general form of a table:
% Fill in the caption in the braces of the \caption{} command. Put the label
% that you will use with \ref{} command in the braces of the \label{} command.
% Insert the column specifiers (l, r, c, d, etc.) in the empty braces of the
% \begin{tabular}{} command.
%
% \begin{table}
% \caption{\label{} }
% \begin{tabular}{}
% \end{tabular}
% \end{table}

% If you have acknowledgments, this puts in the proper section head.
%\begin{acknowledgments}
% Put your acknowledgments here.
%\end{acknowledgments}

% Create the reference section using BibTeX:

%\printbibliography
%\bibliography{mst_xing, mendeley_v2}
\bibliography{mst_xing.bib}
\bibliographystyle{apsrev}
\end{document}
%
% ****** End of file aiptemplate.tex ******
